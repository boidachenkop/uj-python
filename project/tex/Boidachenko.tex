\documentclass[12pt]{report}
\newcommand{\mychapter}[2]{
    \setcounter{chapter}{#1}
    \setcounter{section}{0}
    \chapter*{#2}
    \addcontentsline{toc}{chapter}{#2}
}
\usepackage[a4paper]{geometry}
\usepackage[myheadings]{fullpage}
\usepackage{fancyhdr}
\usepackage{lastpage}
\usepackage{graphicx, wrapfig, subcaption, setspace, booktabs}
\usepackage[T1]{fontenc}
\usepackage[font=small, labelfont=bf]{caption}
% \usepackage{fourier}
\usepackage[protrusion=true, expansion=true]{microtype}
\usepackage[english]{babel}
\usepackage{sectsty}
\usepackage{url, lipsum}
% \usepackage{polski}
\usepackage[utf8]{inputenc}
\setlength\parindent{10pt}
\usepackage{indentfirst}
\usepackage{listings}%для листинга программ
\usepackage{color}%для цвета
\usepackage{caption}

\newcommand{\HRule}[1]{\rule{\linewidth}{#1}}
\doublespacing
\setcounter{tocdepth}{5}
\setcounter{secnumdepth}{5}
%настройки листингов
\definecolor{mygray}{rgb}{0.7,0.7,0.7}%цвет коментариев
\definecolor{myyellow}{rgb}{0.7,0.5,0}
% \lstset{language=C++,
% 		tabsize=2,
% 		showstringspaces=false,
%         breaklines=true, 
%         keywordstyle=\color{blue}\ttfamily, 
%         commentstyle=\color{mygray},
%         stringstyle=\color{myyellow}\ttfamily,}
\usepackage{pythonhighlight}

%-------------------------------------------------------------------------------
% HEADER & FOOTER
%-------------------------------------------------------------------------------
%что бы чаптеры не мешали нумерации секций
\renewcommand{\thesection}{\arabic{section}}
\pagestyle{fancy}
\fancyhf{}
\setlength\headheight{15pt}
\fancyhead[L]{Pavlo Boidachenko}
\fancyhead[R]{Uniwersytet Jagielloński}
\fancyfoot[R]{\thepage}
%-------------------------------------------------------------------------------
% TITLE PAGE
%-------------------------------------------------------------------------------

\begin{document}
\title{ \normalsize \textsc{}
		\\ [2.0cm]
		\HRule{0.5pt} \\
		\LARGE \textbf{{Generowanie labiryntów metodą Kruskala}}
		\HRule{2pt} \\ [0.5cm]
		\normalsize \today \vspace*{5\baselineskip}}

\date{}

\author{
		Pavlo Boidachenko \\ 
		Uniwersytet Jagielloński \\ }

\maketitle
\newpage
%сменил имя с contents на spis tresci
\renewcommand{\contentsname}{Spis treści}
\tableofcontents
\newpage

%-------------------------------------------------------------------------------
% Section title formatting
\sectionfont{\scshape}
%-------------------------------------------------------------------------------
%---------------------------------------------------
%fetures
%---------------------------------------------------
%\begin{figure}[h]
%	\begin{minipage}[h]{0.50\linewidth}
%		\center{
% 		}
% 	\end{minipage}
% 	\begin{minipage}[h]{0.40\linewidth}
%     	\center{
%     	}
% 	\end{minipage}
% \label{ris:image1}
%\end{figure}

% \begin{center}
% \includegraphics[scale=0.9]{photo.png}
% \end{center}
%-------------------------------------------------------------------------------
% BODY
%-------------------------------------------------------------------------------
\section{Opis algorytmu}
\par Algorytm Kruskala jest algorytmem grafowym wyznaczającym minimalne
drzewo rozpinające dla grafu nieskierowanego ważonego. Jak się okazuję 
problem generowania labiryntu łatwo sprowadzić do problemu znanego: 
wyznaczanie minimalnego drzewa rozpinającego. Generowanie labiryntu 
zaczynamy od siatki w której białe komórki są przejściami, 
a czarne ścianami.

\begin{center}
	\includegraphics[scale=0.6]{./pic/grid.png}
	\\\textit{\footnotesize{ Rys 1. 
			Początkowy stan.}}
	\label{ris:image1}
\end{center}

\par Przedstawmy sobie że białe komórki reprezentują węzły grafu, a
 czarne są krawędziami. Przy budowaniu minimalnego drzewa rozpinającego 
za każdym razem jak wybieramy krawędź do naszego drzewa rozpinającego 
usuwamy odpowiednią ścianę w labiryncie. W wyniku otrzymujemy 
labirynt w którym komórki są spójne(z każdej komórki da się przejść 
do każdej innej) i bez cykli.
\par W praktyce generowanie labiryntu przebiega następująco:
\begin{enumerate}
\item Każdą komórkę wkładamy do swojego zbioru.
\item Wybieramy losową ścianę, jeśli komórki które ona rozdziela są w
różnych zbiorach, to złączamy zbiory i usuwamy ścianę.
\item Powtórz krok 2 dopóki są nieodwiedzone ściany.
\end{enumerate}

\newpage

\section{Implementacja}
\subsection{Generowanie}
\par Komórki i ściany zaimplementowałem w jednej strukturze danych 
\textit{WallCell} która ma pole \textit{type} od wartości którego zależy
czy \textit{WallCell} reprezentuje komórkę(wartość 0), czy ścianę
(wartość 0). ID są przydzielane tylko komórkom, żeby można było ich
odróżniać w zbiorach.

\begin{python}
class WallCell:
	def __init__(self, type, ID=None):
		self.id = ID
		self.type = type
		self.row = None
		self.col = None

\end{python}

\par Jądrem implementacji jest metoda \textit{genMaze()}, która wybiera
losową ścianę z listy ścian, szuka sąsiednie komórki dla tej ściany
(może ich być maksymalnie dwie), sprawdza czy wybrane komórki są w
różnych zbiorach i jeżeli są to usuwa ścianę pomiędzy nimi i złącza 
zbiory.

\begin{python}
	def genMaze(self):
		while(len(self.walls) != 0):
			curr = random.choice(self.walls)
			self.walls.remove(curr)
			nb1, nb2 = self.getWallNeighbourCells(curr)
			if self.inDiffSets(nb1, nb2):
				curr.removeWall()
				self.mergeCellSets(nb1, nb2)
				self.frames.append(self.getImage())
\end{python}

\subsection{Wizualizacja}
\par Dla wizualizacji wyników działania programu używam biblioteki PIL
(Python Image Library) - biblioteka Pythona przeznaczona dla pracy z 
grafiką rastrową. Oficjalna strona internetowa: 
\url{http://www.pythonware.com/products/pil/}
\par Biblioteka PIL pozwala na stworzenie animacji w postaci pliku .gif.
Dla tego po każdym złączeniu komórek generuję obrazek i przechowuję go
w liście. Dla przetwarzania macierzy labiryntu, przechowującej obiekty 
klasy \textit{WallCell}, jest metoda \textit{getImage()}, która iteruję
wierszami i kolumnami macierzy i zależnie od typu komórki jest 
rysowany czarny lub biały piksel. Obrazek jest skalowany, żeby
 ładnie się otwierał w programach dla przeglądania grafiki.
 \newpage






\begin{thebibliography}{9}

\bibitem{url1}{Biblioteka PIL \url{http://www.pythonware.com/products/pil/}}
\bibitem{url2}{Wikipedia: Maze generation algorithm \url{https://en.wikipedia.org/wiki/Maze_generation_algorithm}}
\bibitem{url3}{Creating animated GIFs with Pillow \url{http://www.pythoninformer.com/python-libraries/pillow/creating-animated-gif/}}
  

\end{thebibliography}

\end{document}













